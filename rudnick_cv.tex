
\documentclass[11pt]{article}

\usepackage[top=1in, bottom=1in, left=1in, right=1in]{geometry}
%\addtolength{\textheight}{48mm}
%\addtolength{\textwidth}{48mm}
%\addtolength{\voffset}{-24mm}
%\addtolength{\hoffset}{-24mm}

\usepackage{times}
%\usepackage{amsmath}
%\usepackage{amssymb}
%\usepackage{indentfirst}

\newcommand{\HRule}{\rule{\linewidth}{0.7mm}}
%\pagestyle{myheadings}
\markright{ }

\begin{document}

%\setcounter{page}{3}
%\renewcommand{\thepage}{E--\arabic{page}}
\renewcommand{\thepage}{}

\begin{center}
%{\Large Biographical Sketch:} \\
%\vspace{2.5mm}
{\LARGE \bf Gregory H. Rudnick, {\it University of Kansas}}
\end{center}
\vspace{1.0mm}

\begin{flushleft}
%\vspace{-0.3in}
{\large {\bf \textsc{Role in the project}}
\hrulefill} \\
\end{flushleft}

Gregory Rudnick is an expert in ionized gas properties of galaxies, dust emission, molecular gas, distant clusters, and the characterization of galaxy environments.  His knowledge spans many of the areas of the proposal and he will therefore be able to effectively coordinate the various research activities.  He will personally lead investigations into: \textbf{1)} the spatial distribution the ionized gas in intermediate redshift filament, group, and cluster galaxies; \textbf{2)} the environmental dependence of obscured star formation in the same intermediate redshift galaxies; \textbf{3)} and the molecular gas contents of the most distant cluster galaxies.  He is involved in most of the research projects proposed by the team and will therefore be an effective nexus of the research activities.  He will be one of the lead authors on the synthesis papers that the team lists as its primary goal.

On the administrative side, he will organize and run all the team meetings, maintain the e-mail archive, organize and run regular ($\sim$monthly) team telecons, coordinate the writing of team proposals, and maintain the team wiki and public ISSI web page.

\begin{flushleft}
%\vspace{-0.1in}
{\large {\bf \textsc{Fellowships \& Appointments}}
\hrulefill}
\end{flushleft}

%\vspace{-0.3in}
\begin{itemize}
\item \textbf{current position:} Associate Professor of Astronomy, University of Kansas (2013 - present) 
\vspace{-0.2cm}
\item Recipient of an ``Alexander von Humboldt Fellowship'' to conduct
  research at the Max-Planck-Institute for Astronomy in Heidelberg,
  Germany, for the summers of 2012-2014
    %\vspace{-0.08in}
\vspace{-0.2cm}
\item Assistant Professor of Astronomy, University of Kansas (2008 - 2013) 
\vspace{-0.2cm}
\item Leo Goldberg postdoctoral fellow, National Optical Astronomy Observatory (2004 - 2008) 
\vspace{-0.2cm}
\item Postdoctoral researcher, Max-Planck-Institute for Astrophysics (2001 - 2004) 

\end{itemize}


\begin{flushleft}
%\vspace{-0.3in}
{\large {\bf \textsc{Education}}
\hrulefill} \\
\end{flushleft}

\vspace{-0.1in}
\begin{tabular}{l @{\hspace{0.6cm}} l @{\hspace{1cm}} l @{\hspace{1.5cm}} r}
University of Illinois    & Physics         & B.S. & 1996 \\
University of Arizona      & Astronomy       & Ph.D. & 2001 \\
\end{tabular}

%\begin{flushleft}
%\vspace{-0.1in}
%{\large {\bf \textsc{Research Expertise and Interests}}
%\hrulefill}
%\end{flushleft}
%
%\vspace{-0.1in}
%Optical and infrared observational astronomy; The effect of environment on galaxies; The growth of stellar mass in galaxies; Galaxy clusters

\begin{flushleft}
\vspace{-0.1in}
{\large {\bf \textsc{Synergistic Activities and committees}}
\hrulefill}
\end{flushleft}
\begin{itemize}
% \vspace{-0.08in}
 %\vspace{-0.08in}
\item Member of Committee for the Status of Women in Astronomy, a committee of the American Astronomical Society. August 2017 - present
%\vspace{-0.08in}

\item Member of NASA TAC. October 2017
%\vspace{-0.08in}

\item Member of Scientific Organizing Committee for
  ``Early stages of Galaxy Cluster Formation: Mergers, Protoclusters, and Star Formation in Overdense Environments
'', conference on 17-21, July 2017, in Garching, Germany
%\vspace{-0.08in}

\item Member of Scientific Organizing Committee and author of chapter in final report for Kavli foundation sponsored workshop: ``Maximizing Science in the Era of LSST: A Community-based Study of Needed US OIR Capabilities", workshop 2-6, May 2016, in Tucson, USA

\item Member of Scientific Organizing Committee for
  ``In the footsteps of Galaxies'', conference on 7-11 September, 2015, in Soverato, Italy
%\vspace{-0.08in}


%  \vspace{-0.08in}
\item Director of Graduate Studies, Dept. of
  Physics and Astronomy, KU, Fall 2013 - Present
 % \vspace{-0.08in}
\item Member, BigBOSS community science working group
  of the National Optical Astronomy Observatory,
  Jan. 2013 - August 2013

%  \vspace{-0.08in}
\item Member of Scientific Organizing Committee for
  ``Highly Multiplexed Spectroscopy with BigBOSS on
  the Mayall Telescope: An NOAO Community Workshop''
  conference on 13-14 September, 2011, in Tucson
%\vspace{-0.08in}

\item Member of National Optical Astronomy Observatory
  time allocation proposal review committee, May 2008,
  November 2009, May 2010, November 2010, November 2011
 % \vspace{-0.08in}

\item Member of NASA time allocation proposal review committee,
  September 2009, October 2013, April 2014
  %\vspace{-0.08in}

\item Member of Spitzer Space Telescope proposal review committee, April 2007,
  March 2009, December 2012
  %\vspace{-0.08in}

\end{itemize}

\begin{flushleft}
\vspace{-0.1in}
{\large {\bf \textsc{Honors}}
\hrulefill}
\end{flushleft}

\vspace{-0.1in}
One of four recipients of 2016 University Scholarly Achievement Award for excellent scholarly achievement at the University of Kansas

\begin{flushleft}
\vspace{-0.1in}
{\large {\bf \textsc{Related Publications}}
\hrulefill}
\end{flushleft}

\vspace{-0.07in}
\hangindent=1.5cm \hangafter=1 
{\it Determining the Halo Mass Scale where Galaxies Lose Their Gas}, {\bf Rudnick, G.}, Jablonka, P, Moustakas, J., Arag\'on-Salamanca, A., Zaritsky, D., Jaff\'e, Y.~L., De Lucia, G., Desai, V., Halliday, C., Just, D., Milvang-Jensen, B., Poggianti, B., 2017, ApJ, 850, 181

%\vspace{-0.07in}
\hangindent=1.5cm \hangafter=1 
{\it Substantial Molecular Gas Reservoirs from CO(1-0) Observations and Extremely Low Star Formation Efficencies in $z=1.62$ Cluster Galaxies.}, {\bf Rudnick, G.}, Hodge, J., Walter, F., Momcheva, I., Tran, K.-V., Papovich, C., da Cunha, E., Decarlo, R., Saintonge, A., Willmer, C., Lotz, J., Lentati, L., 2017, ApJ, 849, 27
%\vspace{-0.07in}

\hangindent=1.5cm \hangafter=1 
{\it The Ages of Passive Galaxies in a z = 1.62 Protocluster}L ee-Brown, D. B., {\bf Rudnick, G.}, Momcheva, I., Papovich, C., Lotz, J., Tran, K.-V., Henke, B., Willmer, C. N. A.; Brammer, G. B.; Brodwin, M., Dunlop, J., Farrah, D., 2017, ApJ, 844, 43

\hangindent=1.5cm \hangafter=1 
{\it Disc colours in field and cluster spiral galaxies at $0.5<z<0.8$}, Cantale, N., Jablonka, P., Courbin, F., {\bf Rudnick, G.}, Zaritsky, D., Meylan, 
G., Desai, V., De Lucia, G., Arag{\'o}n-Salamanca, A., Poggianti, B.~M., 
Finn, R., and Simard, L.,  2016,  A\&A,  589,  A82 

{%\vspace{-0.07in}
\hangindent=1.5cm \hangafter=1 
{\it A Tale of Dwarfs and Giants: Using a $z=1.62$ Cluster to
  Understand How the Red Sequence Grew Over the Last 9.5 Billion
  years}, {{\bf {Rudnick}, G.}, {Tran}, K.-V., {Papovich}, C.,
  {Momcheva}, I., and {Willmer}, C.}, 2012, ApJ, 755, article id. 14

\hangindent=1.5cm \hangafter=1 
{\it Dust Obscured Star Formation in Intermediate Redshift Clusters}, Finn, R., , Desai, V., {\bf
  Rudnick, G.}, Poggianti, B., Bell, E., and 6 co-authors, 2010, ApJ, 720, 87

\hangindent=1.5cm \hangafter=1 
{\it A Spitzer-selected Galaxy Cluster at $z = 1.62$}, Papovich, C.,
Momcheva, I., Willmer, C.~N.~A., Finkelstein, K.~D., Finkelstein,
S.~L., Tran, K.-V., Brodwin, M., Dunlop, J.~S., Farrah, D., Khan,
S.~A., Lotz, J., McCarthy, P., McLure, R.~J., Rieke, M., {\bf Rudnick, G.},
Sivanandam, S., Pacaud, F., \& Pierre, M.\ 2010, ApJ, 716, 1503-1513

\hangindent=1.5cm \hangafter=1 
{\it The Rest-frame Optical Luminosity Function of Cluster Galaxies at
  $z < 0.8$ and the Assembly of the Cluster Red Sequence}, {\bf Rudnick,
  G.}, {von der Linden}, A., {Pell{\'o}}, R.,
{Arag{\'o}n-Salamanca}, A., and 11 co-authors, 2009, ApJ, 700, 1559

\hangindent=1.5cm \hangafter=1 
{\it Spitzer Mid- to Far-Infrared Flux Densities of Distant
  Galaxies}, {{Papovich}, C., {\bf {Rudnick}, G.}, {Le Floc'h}, E.,
  {van Dokkum}, P.~G., and 7 coauthors,
  2007, ApJ, 668 45

%% \begin{flushleft}
%% \vspace{-0.1in}
%% {\large {\bf \textsc{Other Significant Products}}
%% \hrulefill}
%% \end{flushleft}

%% \vspace{-0.1in}
%% \hangindent=1.5cm \hangafter=1 
%% {\it Star Formation Across the Face of the Eight O'clock Arc},
%%    {Finkelstein}, S. L., {Papovich}, C., {\bf Rudnick, G.}, Egami,
%%    E., Le Floc'h, E., Rieke, M., Rigby, J. R., \& Willmer,
%%    C. N. A., 2009, ApJ, 700, 376
%% %\vspace{0.15cm}

%% \hangindent=1.5cm \hangafter=1 {\it The Number Density and Mass
%%   Density of Star-forming and Quiescent Galaxies at 0.4$<$z$<$2.2},
%% Brammer, G.~B., Whitaker, K.~E., van Dokkum, P.~G., Marchesini, D.,
%% Franx, M., Kriek, M., Labb{\'e}, I., Lee, K.-S., Muzzin, A., Quadri,
%% R.~F., \textbf {Rudnick, G.}, and Williams, R.\ 2011 \ ApJ.\ 739, 24

%% \hangindent=1.5cm \hangafter=1 {\it The Rise of Massive Red Galaxies:
%%   The Color-Magnitude and Color-Stellar Mass Diagrams for $z_{phot} <
%%   2$ from the Multiwavelength Survey by Yale-Chile}, Taylor, E.~N.,
%% Franx, M., van Dokkum, P.~G., Bell, E.~F., Brammer, G.~B., {\bf
%%   Rudnick, G.}, Wuyts, S., Gawiser, E., Lira, P., Urry, C.~M., \& Rix,
%% H.-W.\ 2009, ApJ, 694, 1171-1199


%% %% {\it The Lopsidedness of Present-Day Galaxies: Connections to the
%% %%    Formation of Stars, the Chemical Evolution of Galaxies, and the
%% %%    Growth of Black Holes}, Reichard, T. A., Heckman, T. M., {\bf
%% %%      Rudnick, G.}, Brinchmann, J., Kauffmann, G., Wild, V., 2009, ApJ,
%% %%    691, 1005
%% %\vspace{0.15cm}

%% \hangindent=1.5cm \hangafter=1 
%% {\it Measuring the Average Evolution of Luminous Galaxies at
%%   $z<3$: The Rest-Frame Optical Luminosity Density, Spectral Energy
%%   Distribution, and Stellar Mass Density}, {{\bf Rudnick, G.},,
%%   {Labb{\'e}}, I., {F{\"o}rster Schreiber}, N.~M., {Wuyts}, S.,
%%   and 10 coauthors, 2006, ApJ, 650, 624
%% %\vspace{0.15cm}

%% \hangindent=1.5cm \hangafter=1 
%% {\it The Rest-Frame Optical Luminosity Density, Color, and
%%   Stellar Mass Density of the Universe from $z=0$ to $z=3$}, {\bf
%%   Rudnick, G.}, Rix, H.-W., Franx, M., and 11 coauthors, 2003, ApJ, {\bf 599}, 847-864.



\end{document}











